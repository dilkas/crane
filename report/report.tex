\documentclass[12pt]{report}

\usepackage{amsmath}
\usepackage{graphicx}
\usepackage{listings}
\usepackage{hyperref}
\usepackage{algorithm2e}
\usepackage[a4paper, total={7in,9in}, headheight=14pt]{geometry}
\usepackage{minted}

\lstset{breaklines=true}
\lstset{language=C++,
        basicstyle=\ttfamily,
        stringstyle=\color{red},
        commentstyle=\color{green},
        breaklines=true,
        showstringspaces=false}

\title{Combined Report of Weeks 1 to 9}
\author{Ananth Krishna Kidambi, Guramrit Singh}
\begin{document}
    \maketitle
    \tableofcontents
    \chapter*{Week 9}
    \addcontentsline{toc}{chapter}{Week 9}
    \section*{Code Implementation}
    \subsection*{Example}
    Consider the set of equations - 
    \begin{align*}
        f0(x0,x1)&=\sum_{x2=0}{x1}((-1)^{x1-x2}\cdot\\ & \binom{x1}{x2}\cdot \sum_{x3=0}{x0}[(-1)^{x0-x3}\cdot \binom{x0}{x3}\cdot f1(x2,x3))\\
        f1(x2,x3)&=\binom{x3}{0}\cdot f1(-1+x2,x3-0)+\binom{x3}{1}\cdot f1(-1+x2,x3-1)\\
        f1(0,x3)&=1\\
        f1(x2,0)&=1\\
    \end{align*}
    The corresponding code generated is - 
    \begin{minted}[tabsize=4,linenos]{cpp}
#include <iostream>
#include <string>
#include <vector>
#include <cmath>
#include <gmpxx.h>

class cache_elem{
public : 
	mpz_class n;
	cache_elem(mpz_class x) : n{x} {}
	cache_elem() : n{-1} {}
};

template <class T> T& get_elem(std::vector<T>& a, size_t n){
	if (n >= a.size()){
		a.resize(n+1);
	}
	return a.at(n);
}

mpz_class Binomial(unsigned int n, unsigned int r){
	mpz_t ans;
	mpz_init(ans);
	mpz_bin_uiui(ans, n, r);
	return mpz_class{ans};
}

mpz_class power(mpz_class x, unsigned int y){
	mpz_t ans;
	mpz_init(ans);
	mpz_pow_ui(ans, x.get_mpz_t(), y);
	return mpz_class{ans};
}

std::vector<cache_elem> f0_cache;
std::vector<std::vector<cache_elem>> f1_cache;

mpz_class f0(unsigned int x0);
mpz_class f1(unsigned int x1, unsigned int x2);
mpz_class f1_0x(unsigned int x2);
mpz_class f1_x0(unsigned int x1);

mpz_class f0(unsigned int x0){
	mpz_class& stored_val = get_elem(f0_cache, x0).n;
	if (stored_val != -1)
		return stored_val;
	if (x0 >= 0){
		mpz_class ret_val = ([x0](){mpz_class sum{0}; for (unsigned x1 = 0; x1 <= x0; x1++){ sum += (Binomial(x0,x1)*(power(-1,x0-x1)*([x0,x1](){mpz_class sum{0}; for (unsigned x2 = 0; x2 <= x0; x2++){ sum += (Binomial(x0,x2)*(power(-1,x0-x2)*f1(x1,x2)));} return sum;})()));} return sum;})();
		get_elem(f0_cache, x0).n = ret_val;
		return ret_val;
	}
	exit(1);
	return -1;
}
mpz_class f1(unsigned int x1, unsigned int x2){
	mpz_class& stored_val = get_elem(get_elem(f1_cache, x1), x2).n;
	if (stored_val != -1)
		return stored_val;
	if (x1 >= 1 && x2 >= 1){
		mpz_class ret_val = (Binomial(x2,0)*f1(x1-1,x2-0))+(Binomial(x2,1)*f1(x1-1,x2-1));
		get_elem(get_elem(f1_cache, x1), x2).n = ret_val;
		return ret_val;
	}
	else if (x1 == 0){
		return  f1_0x( x2);
	}
	else if (x2 == 0){
		return  f1_x0( x1);
	}
	exit(1);
	return -1;
}
mpz_class f1_0x(unsigned int x2){
	mpz_class& stored_val = get_elem(get_elem(f1_cache, 0), x2).n;
	if (stored_val != -1)
		return stored_val;
	if (x2 >= 0){
		mpz_class ret_val = 1;
		get_elem(get_elem(f1_cache, 0), x2).n = ret_val;
		return ret_val;
	}
	exit(1);
	return -1;
}
mpz_class f1_x0(unsigned int x1){
	mpz_class& stored_val = get_elem(get_elem(f1_cache, x1), 0).n;
	if (stored_val != -1)
		return stored_val;
	if (x1 >= 0){
		mpz_class ret_val = 1;
		get_elem(get_elem(f1_cache, x1), 0).n = ret_val;
		return ret_val;
	}
	exit(1);
	return -1;
}

int main(){
	std::cout << f0(2048) << std::endl;
}
    \end{minted}
    \chapter*{Week 8}
    \addcontentsline{toc}{chapter}{Week 8}
    \section*{Code Implementation}
    \subsection*{Example}
    Consider the set of equations - 
    \begin{align*}
        f0(x0,x1)&=\sum_{x2=0}{x1}((-1)^(x1-x2)\cdot\\ & Binomial(x1,x2)\cdot \sum_{x3=0}{x0}[(-1)^(x0-x3)\cdot Binomial(x0,x3)\cdot f1(x2,x3))\\
        f1[x2,x3]&=Binomial(x3,0)\cdot f1(-1+x2,x3-0)+Binomial(x3,1)\cdot f1[-1+x2,x3-1]\\
        f1[0,x3]&=1\\
        f1[x2,0]&=1\\
    \end{align*}
    The corresponding code generated is - 
    \begin{minted}[tabsize=4,linenos]{cpp}
#include <iostream>
#include <string>
#include <vector>
#include <cmath>

class cache_elem{
public : 
	int n;
	cache_elem(int x) : n{x} {}
	cache_elem() : n{-1} {}
};

template <class T> T& get_elem(std::vector<T>& a, size_t n){
	if (n >= a.size()){
		a.resize(n+1);
	}
	return a.at(n);
}

int Binomial(int n, int r){
	return round(std::tgamma(n+1)/(std::tgamma(r+1)*std::tgamma(n-r+1)));
}

int power(int x, int y){
	return round(pow(x, y));
}

std::vector<std::vector<cache_elem>> f0_cache;
std::vector<std::vector<cache_elem>> f1_cache;

int f0(int x0, int x1);
int f1(int x2, int x3);
int f1_0x(int x3);
int f1_x0(int x2);

int f0(int x0, int x1){
	int stored_val = get_elem(get_elem(f0_cache, x0), x1).n;
	if (stored_val != -1)
		return stored_val;
	if (x0 >= 0 && x1 >= 0){
		int ret_val = ([x0,x1](){int sum{0}; for (unsigned x2 = 0; x2 <= x1; x2++){ sum += ((power(-1,x1-x2)*Binomial(x1,x2))*([x0,x1,x2](){int sum{0}; for (unsigned x3 = 0; x3 <= x0; x3++){ sum += ((power(-1,x0-x3)*Binomial(x0,x3))*f1(x2,x3));} return sum;})());} return sum;})();
		get_elem(get_elem(f0_cache, x0), x1).n = ret_val;
		return ret_val;
	}
	return -1;
}
int f1(int x2, int x3){
	int stored_val = get_elem(get_elem(f1_cache, x2), x3).n;
	if (stored_val != -1)
		return stored_val;
	if (x2 >= 1 && x3 >= 1){
		int ret_val = (Binomial(x3,0)*f1(-1+x2,x3-0))+(Binomial(x3,1)*f1(-1+x2,x3-1));
		get_elem(get_elem(f1_cache, x2), x3).n = ret_val;
		return ret_val;
	}
	else if (x2 == 0){
		return  f1_0x( x3);
	}
	else if (x3 == 0){
		return  f1_x0( x2);
	}
	return -1;
}
int f1_0x(int x3){
	int stored_val = get_elem(get_elem(f1_cache, 0), x3).n;
	if (stored_val != -1)
		return stored_val;
	if (x3 >= 0){
		int ret_val = 1;
		get_elem(get_elem(f1_cache, 0), x3).n = ret_val;
		return ret_val;
	}
	return -1;
}
int f1_x0(int x2){
	int stored_val = get_elem(get_elem(f1_cache, x2), 0).n;
	if (stored_val != -1)
		return stored_val;
	if (x2 >= 0){
		int ret_val = 1;
		get_elem(get_elem(f1_cache, x2), 0).n = ret_val;
		return ret_val;
	}
	return -1;
}

int main(){
	std::cout << f0(3,3) << std::endl;
}
    \end{minted}
\end{document}
